\pagenumbering{arabic}
\setcounter{page}{1}
\chapter{MỞ ĐẦU}
\section{Lý do chọn đề tài}

Trong thế giới kinh doanh ngày nay, sự cạnh tranh ngày càng gay gắt. Các doanh nghiệp cần phải luôn nhạy bén với thị trường để có thể đưa ra những quyết định kinh doanh đúng đắn,từ đó nâng cao hiệu quả hoạt động và tăng khả năng cạnh tranh. Một trong những quyết định quan trọng nhất là quyết định về sản xuất và cung ứng. Để đưa ra quyết định này một cách hiệu quả, doanh nghiệp cần có một dự báo nhu cầu sản phẩm chính xác.

Dự báo nhu cầu sản phẩm là quá trình ước lượng nhu cầu của khách hàng đối với một sản phẩm hoặc dịch vụ trong tương lai. Dự báo này dựa trên các yếu tố như: lịch sử bán hàng, xu hướng thị trường, các yếu tố kinh tế, chính trị, xã hội, v.v.

Những mô hình dự báo theo hướng học sâu thường được sử dụng có thể kể đến như mạng nơ-ron nhân tạo (Artificial neural network – ANN), bộ nhớ dài ngắn hạn (Long short-term memory – LSTM), \dots. Tuy nhiên, lớp mạng trên yêu cầu lượng lớn dữ liệu lịch sử trong quá trình huấn luyện để đạt được hiệu quả tốt nhất. Để đáp ứng với điều kiện thực tế thiếu thốn dữ liệu, các mô hình được chọn trong báo cáo này bao gồm: mô hình VAR và mô hình làm mượt Holt-Winters. Báo cáo bao gồm trình bày lí thuyết và thực nghiệm mô hình trên bộ dữ liệu thực tế.

\section{Bài toán, đối tượng và phạm vi nghiên cứu}

Bài toán: Dự báo kế hoạch sản phẩm A, B trong các quý tiếp theo của công ty ATS.

Đối tượng nghiên cứu: Dữ liệu lịch sử kế hoạch sản phầm A, B của công ty ATS và công ty đối thủ.

Phạm vi nghiên cứu: Từ quý 1 năm 2020 đến quý 4 năm 2023.