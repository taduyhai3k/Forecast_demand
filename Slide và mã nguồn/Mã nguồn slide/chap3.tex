\begin{frame}{Mô hình làm mượt Holt-Winters}
    Phương pháp Holt-Winters là một phương pháp dự báo chuỗi thời gian
mạnh mẽ được thiết kế để xử lý các chuỗi có tính xu hướng và tính mùa.

Trong phần này nhóm tác giả trình bày phương pháp Holt-Winters cộng tính và biến thể để sử dụng với chuỗi thời gian đa biến. 
\end{frame}

\begin{frame}{Mô hình làm mượt Holt-Winters}
       Xét chuỗi thời gian đơn biến $\textbf{y} = \left\{y_1, \dots, y_t\right\}$ với tính mùa có chu kỳ $p$. 
       
       Các quan sát này được biểu diễn dưới dạng $y_i = a_{i-1} + b_{i-1} + c_{i-1}, \ i= \overline{1, t}$, trong đó $a_i$ là thành phần mức độ, $b_i$ là phần xu thế và $c_i$ là phần mùa.
\end{frame}

\begin{frame}{Mô hình làm mượt Holt-Winters}
    Phương pháp cộng tính có $3$ phương trình làm mượt và một phương trình dự báo như sau:
    \begin{align} 
        &a_i = \alpha\left(y_i - c_{i-p}\right) + \left(1-\alpha\right)\left(a_{i-1} + b_{i-1}\right),\label{eq:3.1} \\
        &b_i = \beta\left(a_i - a_{i-1}\right) + \left(1 - \beta\right)b_{i-1}, \label{eq:3.2}\\
        &c_i = \gamma\left(y_i - a_{i-1} - b_{i-1}\right) + \left(1-\gamma\right)c_{i-p},\label{eq:3.3} \\
        &\hat{y}_{t+h \vert t} = a_t + h b_t + c_{t+h - p\left(k+1\right)}. \label{eq:3.4}
    \end{align}
    
\end{frame}

\begin{frame}{Mô hình làm mượt Holt-Winters}
    Bermúdez và các cộng sự \footnote{Bermúdez. J. D., Segura, J. V., \& Vercher, E. , Holt–Winters forecasting: an alternative formulation applied to UK air passenger data, \textit{Journal of Applied Statistics}, 34(9), 1075-1090, 2007.} đề xuất sử dụng mô hình hồi quy để ước lượng các tham số $\alpha, \beta, \gamma$ và giá trị ban đầu để thực hiện phương pháp cộng tính. Cụ thể các biến ngẫu nhiên $Y_i$ được biểu diễn dưới dạng
    \begin{align}
        Y_i = a_{i-1} + b_{i-1} + c_{i-1} +  \epsilon_i. \label{eq:3.5}
    \end{align}
    Trong đó $\epsilon_i \sim \mathcal{N}\left(0, \sigma^2\right)$.
\end{frame}


\begin{frame}{Mô hình làm mượt Holt-Winters}
    Kết hợp \ref{eq:3.5}, \ref{eq:3.1}, \ref{eq:3.2}, \ref{eq:3.3} thu được biểu diễn:

    \begin{align*}
        &Y_1 = a_0 + b_0 + c_{1-p} + \epsilon_1  \\
        &Y_2 = a_0 + 2b_0 + c_{2-p} + \alpha\left(1+\beta\right)\epsilon_1 +  \epsilon_2  \\
        & \vdots \\
        & Y_{p+1} = a_0 + \left(p + 1\right)b_0 + c_{1-p} + \gamma \epsilon_1 + \alpha \sum_{r= 1}^p \left(1 + \beta\left(p+1-r\right)\right)\epsilon_r + \epsilon_{p+1} \\
        &\dots
    \end{align*}
\end{frame}


\begin{frame}{Mô hình làm mượt Holt-Winters}
      Từ biểu diễn trên, tổng hợp lại thu được biểu diễn dạng ma trận
    \begin{align}
        \textbf{Y} = M \psi + L \epsilon         
    \end{align}
    
    Trong đó $\psi = \left(b_0, c_{1-p}, c_{2-p}, \dots, c_0\right)', a_0 = -b_0$ là véc tơ điều kiện ban đầu, $M$ là ma trận cỡ $t \times \left(p+1\right)$ với cột đầu tiên là $\left(0,1, \dots,t-1 \right)$ và $p$ cột còn lại được tạo bởi các ma trận đơn vị cỡ $p \times p$ xếp từ trên xuống dưới đủ $t$ hàng.  
\end{frame}

\begin{frame}{Mô hình làm mượt Holt-Winters}
     Ma trận $L$ có biểu diễn
    \begin{align}
        L = \begin{bmatrix}
        1 & 0 & 0 & \dots & 0  & 0\\
        l_2 & 1 & 0 & \dots & 0  & 0\\
        l_3 & l_2 & 1 & \dots & 0  &0\\
        \vdots& \vdots& \vdots& \ddots&\vdots &\vdots\\
        l_{n-1} & l_{n-2} & l_{n-3} & \dots& 1 & 0\\
        l_n& l_{n-1}& l_{n-2}&\dots&l_2&1
        \end{bmatrix}
    \end{align}
    
    với $l_i = \alpha\left(1 + \left(i-1\right)\beta\right) + \gamma\left(i = 1 \ \text{mod} \ p \right)$. $\epsilon = \left(\epsilon_1, \dots, \epsilon_t\right)$ là véc tơ nhiễu ngẫu nhiên độc lập cùng phân phối chuẩn.
\end{frame}


\begin{frame}{Mô hình làm mượt Holt-Winters}
    Hàm Log-Likelihood của $\textbf{Y}$ lúc này bằng
    \begin{align}
        -\dfrac{t}{2}\ln\left(\sigma^2\right) - \dfrac{1}{2\sigma^2} \left(\textbf{Y} - M \psi\right)'\left(LL'\right)^{-1}\left(\textbf{Y} - M \psi\right).
    \end{align}
    Các tham số $\alpha, \beta, \gamma$ được xác định bằng cực đại hàm hợp lý trên, tương đương với cực tiểu hóa
    \begin{align}
        \min_{\alpha, \beta, \gamma} \left(L^{-1} \textbf{Y}\right)' \left(I - P_X\right) L^{-1} \textbf{Y}.
    \end{align}
    
    Trong đó $X = L^{-1}M$ và $P_X = X \left(X'X\right)^{-1}X'$.
\end{frame}

\begin{frame}{Mô hình làm mượt Holt-Winters}
    Sau khi thu được $\alpha, \beta, \gamma$, ta sẽ tính lại $\hat{L}, \hat{X}$, từ đó ước lượng được véc tơ điều kiện ban đầu
    \begin{align}
        \hat{\psi} = \left(\hat{X}'\hat{X}\right)^{-1} \hat{X}'\hat{L}^{-1} \textbf{Y}.
    \end{align}
    
    Với bộ điều kiện ban đầu $\psi$, ta đã có thể sử dụng phương pháp cộng tính để tính các tham số và đưa ra dự báo. Lưu ý rằng mô hình trên hoạt động với chuỗi thời gian đơn biến.
\end{frame}

\begin{frame}{Mô hình làm mượt Holt-Winters}
    Xét chuỗi thời gian đa biến $\mathcal{Y} = \left(\textbf{Y}_1, \dots, \textbf{Y}_n\right) $ với $\textbf{Y}_i = \left(Y_{1i}, \dots, Y_{ti}\right)$. Để mô hình hồi quy trên áp dụng được, nhóm tác giả coi các véc tơ $\textbf{Y}_i$ đều có biểu diễn
    \begin{align}
        \textbf{Y}_i = M \psi + L \epsilon.  
    \end{align}
    
    Hay các chuỗi đều có chung bộ tham số $\alpha, \beta, \gamma$.
\end{frame}

\begin{frame}{Mô hình làm mượt Holt-Winters}
    Các tham số này được tìm bằng cách cực tiểu hóa
    \begin{align}
        \min_{\alpha, \beta, \gamma} \sum_{i=1}^n \left(L^{-1} \textbf{Y}_i\right)' \left(I - P_X\right) L^{-1} \textbf{Y}_i.        
    \end{align}
    
    Các ước lượng điều kiện ban đầu thu được bởi
    \begin{align}
        \hat{\psi}_i = \left(\hat{X}'\hat{X}\right)^{-1} \hat{X}'\hat{L}^{-1} \textbf{Y}_i.        
    \end{align}
\end{frame}

\begin{frame}{Mô hình làm mượt Holt-Winters}
        Ngoài ra nhóm tác giả còn đề xuất cách tìm $\alpha, \beta, \gamma$ thông qua cực tiểu hàm bình phương sai số
    \begin{align}
        \min_{\alpha, \beta,\gamma} \dfrac{\sum_{i=1}^n w_i \sum_{j=1}^t \left(Y_{ji} - \hat{Y}_{ji}\right)^2}{\sum_{i=1}^n w_i}.
    \end{align}
    Với $\hat{Y}_{ji} = a_{j-1, i} + b_{j-1,i} + c_{j-1,i}$.
\end{frame}