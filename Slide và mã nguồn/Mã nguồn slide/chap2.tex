\begin{frame}{Mô hình VAR}
    Ta có chuỗi thời gian đa biến K chiều T quan sát $y_1,\dots,y_T$ với $y_t = (y_{1t}, \dots, y_{Kt})$ là quá trình dừng và công thức chung:
\begin{equation}
    y_t = \nu + A_1 y_{t-1} + \dots + A_p y_{t-p} + u_t. \label{ep:2.1}
\end{equation}

Với $\nu = (\nu_1, \dots, \nu_K)$ là vector hằng số $K \times 1$, $A_i$ là ma trận hệ số $K \times K$ và $u_t$ là nhiễu trắng với ma trận hiệp phương sai $\Sigma_u$.
\end{frame}

\begin{frame}{Phương pháp Ordinary Least Squares (OLS)}
    Phương pháp được trình bày bởi \text{Helmut L\"utkepohl} \footnote{\text{Helmut L\"utkepohl}, \textit{New Introduction to Multiple Time Series Analysis}, Springer Berlin, Heidelberg, 2005}. Ta định nghĩa:
    \begin{align*}
    & Y := (y_1, \dots, y_T) & (K \times T), \\    
    & B := (\nu, A_1, \dots, A_p) & (K \times (Kp + 1)), \\   
    & Z_t := [1, y_t, \dots, y_{t-p+1}]' &  ((Kp + 1) \times 1), \\
    & Z := (Z_0, \dots, Z_{T-1}) & ((Kp + 1) \times T), \\
    & U := (u_1,\dots, u_T) & (K \times T), \\
    & \textbf{y} := \text{vec}(Y) & (KT \times 1), \\
    & \beta := \text{vec}(B) & ((K^2p+K) \times 1), \\
    & \textbf{b} := \text{vec}(B') & ((K^2p+K) \times 1), \\
    & \textbf{u} := \text{vec}(U) & (KT \times 1). \\
\end{align*}
    
\end{frame}

\begin{frame}{Phương pháp Ordinary Least Squares (OLS)}
    Khi đó, công thức \ref{ep:2.1} được viết lại thành

\begin{equation}
    Y = BZ + U
\end{equation}

Khi đó, ta có: 
\begin{align}
        && \text{vec}(Y) &= \text{vec}(BZ) +\text{vec}(U) \nonumber \\  
       \Leftrightarrow && \text{vec}(Y) & = (Z' \otimes I_K) \text{vec}(B) + \text{vec}(U) \nonumber \\ 
        \Leftrightarrow && \textbf{y}  &= (Z' \otimes I_K) \beta + \textbf{u}. \label{eq:2.3}
\end{align}

\end{frame}

\begin{frame}{Phương pháp Ordinary Least Squares (OLS)}
Ta cần cực tiểu hàm sau để tìm được ước lượng cho $\beta$:

\begin{equation}
    S(\beta) = \textbf{u}' (I_T \otimes \Sigma_u)^{-1} \textbf{u}.
\end{equation}

Ta thu được nghiệm tối ưu:

\begin{equation}
    \hat{\beta} = ((ZZ')^{-1}Z \otimes I_K )\textbf{y}.
\end{equation}
\end{frame}

\begin{frame}{Phương pháp  Durbin-Levinson}
    Whittle (1963) và Robinson (1963) đã khái quát thuật toán Durbin - Levison cho trường hợp chuỗi thời gian đa biến\footnote{Jones. R. H.,  Multivariate autoregression estimation using residuals. \textit{In Applied Time Series Analysis I} (pp. 139-162). Academic Press, 1978.}.

    Ta có ma trận hiệp phương sai kích thước $K \times K$ với độ trễ $j$:
\begin{equation}
    \hat{R_j} = \frac{1}{n} \sum_{t = 1} ^{n-j} y_{t+j} y_t '.
\end{equation}
\end{frame}

\begin{frame}{Phương pháp  Durbin-Levinson}
    Chuỗi AR bậc $p$ có dự đoán tiến:

\begin{equation}
    \hat{y_t}^{(f)} = \sum_{k = 1}^p A_k^{(p)} y_{t-k},
\end{equation}

và dự đoán lùi:

\begin{equation}
    \hat{y_t}^{(b)} = \sum_{k=1}^p B_k^{(p)} y_{t+k}, 
\end{equation}

với $A_k^{(p)}$ và $B_k^{(p)}$ là ma trận $K \times K$. 
\end{frame}

\begin{frame}{Phương pháp  Durbin-Levinson}
    Giá trị khởi tạo:

\begin{equation}
    S_0^{(f)} = S_0^{(b)} = R_0
\end{equation}

Bước đầu tiên:

\begin{align}
    & A_1^{(1)} = R_1 \left[ S_0 ^{(b)} \right]^{-1} \\
    & B_1^{(1)} = R_{-1} \left[ S_0 ^{(f)} \right]^{-1}
\end{align}


với 
\[
    R_{-1} = R_1'
\]
\end{frame}

\begin{frame}{Phương pháp  Durbin-Levinson}
    Công thức chung:

\begin{align}
    & A_p^{(p)} = (R_p - \sum_{k=1}^{p-1} A_k^{(p-1)} R_{(p-k)} ) \left[ S_{(p-1)}^{(b)}\right]^{-1} \\
    & B_p^{(p)} = (R_p' - \sum_{k=1}^{p-1} B_k^{(p-1)} R'_{(p-k)} ) \left[ S_{(p-1)}^{(f)}\right]^{-1}
\end{align}

Công thức cập nhật ma trận:

\begin{align}
    & A_k^{(p)} = A_k^{(p-1)} - A_p^{(p)} B_{p-k}^{(p-1)} \\
     & B_k^{(p)} = B_k^{(p-1)} - B_p^{(p)} A_{p-k}^{(p-1)}
\end{align}
\end{frame}

\begin{frame}{Phương pháp  Durbin-Levinson}
    Dự báo 1 bước tiến và lùi của ma trận hiệp phương sai nhiễu:

\begin{align}
    & S_p^{(f)} = (I - A_p^{(p)} B_p^(p)) S_{p-1} ^{(f)} \\
    & S_p^{(b)} = (I - B_p^{(p)} A_p^(p)) S_{p-1} ^{(b)}
\end{align}
\end{frame}


